Ez a szakdolgozat betekintést nyújt az új generációs Thread hálózatok felépí-tésébe, működtetésébe és a Thread mesh hálózatokban rejlő potenciális alkalmazási lehetőség-ekre. A dolgozatban a nemrégiben megjelent Nordic nRF52840-dongle modulok és saját nyomtatott áramkörök segítségével mutatom be a Thread kommunikációt több modul és egy átjáró között. Az ilyen Thread hálózatok legnagyobb előnye a vetélytársakkal szemben a kiváló skálázhatóság, mesh topológia miatt a kiemelkedően jó területi lefedettség és a megfelelően gyors adatátvitel. A Thread hálózatok egyik kiemelkedő alkalmazási területe az okosotthon(ok). Ezen felhasználási területen sokféle Thread hálózat kompatibilis eszközt kapcsolunk egy nagy egységes mesh hálózatba, és a  Thread hálózati végpontokon keletkezett szenzor/beavatkozó modul adatokat egy egységes adatbázisban tárolhatjuk, amely hozzásegít az okosotthonok szabályozási és monitoring rendszerének hatékony kialakításához. Dolgozatomban felépítek egy Thread alapú okosotthon platformot (hardver és szoftver szempontból is), amelyben egy központi egységet (Border Router) és több szenzorral ellátott végpontot (Endpoint) használok fel. Az adatok megjelenítéséhez egy grafikus felhasználói interfészt is készítettem, amely segítségével az egyes szenzoradatok megjeleníthetők, illetve az eszközök fel és lecsatlakoztathatók.