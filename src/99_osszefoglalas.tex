\section{Summary}
In this bachelor's thesis, I presented the potential of Thread networks for application in smart homes as part of my research. I have described and presented in detail the Thread protocol implemented by Google, OpenThread and the roles of the tools involved. I also introduced a server-client protocol that can be implemented in embedded environments, in devices with fewer resources, with little overhead. I also introduced the Zephyr Project's revolutionary solution for processors of different architectures, which makes it easy, fast and platform-independent to implement our ideas in software form in practice and to create programs for different hardware with the same capabilities without modifying the code. It currently supports more than 350 development boards and includes different layers and protocols to easily build IoT devices. And as part of my own work, I've created a concept to make the capabilities of Thread available to consumers. This concept includes a database model and an easy-to-use GUI that anyone can use to manage their own network from anywhere. The concept also includes a ready-to-use, out-of-the-box gateway that allows users to create their own network and connect it to their home network and the Internet. I designed a device that can be hidden invisibly in the wall at the point of use, providing the backbone of the network and controlling the lighting in a room. I built 3 devices and used three of these devices in an apartment to create my own Thread network, proving the concept works. I also designed a very small device to collect environmental data of my room, which only requires a single coin-cell battery and is able to operate for years on it. 